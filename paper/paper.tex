\PassOptionsToPackage{table}{xcolor}
\documentclass[review,onefignum,onetabnum,a4paper]{siamart190516}
\pdfoutput=1
\overfullrule=0pt

%------------------------------------------------------------------------------
% Header
%------------------------------------------------------------------------------
\usepackage[utf8]{inputenc}
\usepackage[english]{babel}
\usepackage{amsmath, amssymb, gensymb, physics, upgreek, siunitx, bbm}
\usepackage{tikz, pgfplots, graphicx, epstopdf, float, hyperref, xspace}
\usepackage[caption=false]{subfig}
\usepackage{enumitem, verbatim, listings}
\usepackage{mathtools, stmaryrd}
\usepackage{booktabs, array, ragged2e}
\usepackage[export]{adjustbox}
\usepackage{cleveref}

\sisetup{table-number-alignment=center, exponent-product=\times}
\newcommand{\crefrangeconjunction}{--}

\usetikzlibrary{trees, cd, babel}
\graphicspath{{./figures}}
\epstopdfsetup{outdir=./}

\lstset{basicstyle=\footnotesize\ttfamily}

\definecolor{paired1}{HTML}{a6cee3}
\definecolor{paired2}{HTML}{1f78b4}
\definecolor{paired3}{HTML}{b2df8a}
\definecolor{paired4}{HTML}{33a02c}
\definecolor{paired5}{HTML}{fb9a99}
\definecolor{paired6}{HTML}{e31a1c}
\definecolor{paired7}{HTML}{fdbf6f}
\definecolor{paired8}{HTML}{ff7f00}
\definecolor{paired9}{HTML}{cab2d6}
\definecolor{paired10}{HTML}{6a3d9a}

\hypersetup{
    colorlinks,
    linkcolor={red!50!black},
    citecolor={blue!50!black},
    urlcolor={blue!80!black}
}

\pgfplotsset{
  log x ticks with fixed point/.style={
      xticklabel={
        \pgfkeys{/pgf/fpu=true}
        \pgfmathparse{exp(\tick)}%
        \pgfmathprintnumber[fixed relative, precision=3]{\pgfmathresult}
        \pgfkeys{/pgf/fpu=false}
      }
  },
  log y ticks with fixed point/.style={
      yticklabel={
        \pgfkeys{/pgf/fpu=true}
        \pgfmathparse{exp(\tick)}%
        \pgfmathprintnumber[fixed relative, precision=3]{\pgfmathresult}
        \pgfkeys{/pgf/fpu=false}
      }
  }
}

% Classic symbols
\newcommand{\reals}{\mathbb{R}}
\newcommand{\integers}{\mathbb{Z}}
\newcommand{\dx}{\,\d\mathbf{x}}
\newcommand{\md}{\,\d}
\newcommand{\bigo}[1]{\mathcal{O}(#1)}
\DeclareMathOperator*{\argmin}{arg\,min}


\let\grad\undefined
\let\curl\undefined
\let\div\undefined
\let\nullop\undefined
\let\d\undefined
\let\tr\undefined
\DeclareMathOperator{\grad}{grad}
\DeclareMathOperator{\curl}{curl}
\DeclareMathOperator{\div}{div}
\DeclareMathOperator{\nullop}{null}
\DeclareMathOperator{\d}{d}
\DeclareMathOperator{\tr}{tr}
\DeclareMathOperator{\Exists}{\exists}
\DeclareMathOperator{\Forall}{\forall}

\DeclareMathOperator{\refgrad}{\widehat{\grad}}
\DeclareMathOperator{\refcurl}{\widehat{\curl}}
\DeclareMathOperator{\refdiv}{\widehat{\div}}

\newcommand{\Hgrad}{H(\grad)}
\newcommand{\Hcurl}{H(\curl)}
\newcommand{\Hdiv}{H(\div)}
\newcommand{\Ltwo}{L^2}


\newcommand{\mesh}{\mathcal{T}_h}
\newcommand{\skeleton}{\mathcal{E}_h}
\newcommand{\refline}{\hat{\mathcal{I}}}
\newcommand{\jac}{\mathrm{D}F}
\newcommand{\pullback}{\mathcal{F}}
\newcommand{\betah}{\tilde{\beta}}

% Vector notation
\renewcommand{\vec}[1]{\mathbf{#1}}
\newcommand{\uvec}[1]{\mathbf{\hat{#1}}}
\newcommand{\conj}[1]{{\overline{#1}}}
\DeclareMathOperator{\blockdiag}{blockdiag}
\DeclareMathOperator{\diag}{diag}
\DeclareMathOperator{\spn}{span}
\DeclareMathOperator{\vecop}{vec}
\newcommand{\eye}{{\mathbb{I}}}
\newcommand{\kron}{\otimes}
\newcommand{\jump}[1]{\left\llbracket #1 \right\rrbracket}
\newcommand{\avg}[1]{\left\{ #1 \right\}}
\newcommand{\broken}[1]{\tilde{#1}}


\newcommand{\be}{\vec{e}}
\newcommand{\bu}{\vec{u}}
\newcommand{\bn}{\vec{n}}
\newcommand{\bt}{\vec{t}}
\newcommand{\bm}{\vec{m}}
\newcommand{\bv}{\vec{v}}
\renewcommand{\bf}{\vec{f}}
\newcommand{\bq}{\vec{q}}
\newcommand{\br}{\vec{r}}
\newcommand{\bx}{\vec{x}}
\newcommand{\bw}{\vec{w}}
\newcommand{\by}{\vec{y}}
\newcommand{\bz}{\vec{z}}
\newcommand{\bB}{\vec{B}}
\newcommand{\bC}{\vec{C}}
\newcommand{\bE}{\vec{E}}
\newcommand{\bF}{\vec{F}}
\newcommand{\bJ}{\vec{J}}
\newcommand{\bP}{\vec{P}}
\newcommand{\bT}{\vec{T}}
\newcommand{\bX}{\vec{X}}

\newcommand{\ub}{\underline{b}}
\newcommand{\uc}{\underline{c}}
\newcommand{\ue}{\underline{e}}
\newcommand{\uf}{\underline{f}}
\newcommand{\ug}{\underline{g}}
\newcommand{\up}{\underline{p}}
\newcommand{\uq}{\underline{q}}
\newcommand{\ur}{\underline{r}}
\newcommand{\uu}{\underline{u}}
\newcommand{\uv}{\underline{v}}
\newcommand{\uw}{\underline{w}}
\newcommand{\ux}{\underline{x}}
\newcommand{\uy}{\underline{y}}
\newcommand{\uz}{\underline{z}}
\newcommand{\ut}{\underline{t}}
\newcommand{\ulambda}{\underline{\lambda}}

\newcommand{\uB}{\underline{B}}
\newcommand{\ubb}{\underline{\vec{b}}}
\newcommand{\ubq}{\underline{\vec{q}}}
\newcommand{\ubu}{\underline{\vec{u}}}
\newcommand{\ubv}{\underline{\vec{v}}}
\newcommand{\ubf}{\underline{\vec{f}}}
\newcommand{\ubg}{\underline{\vec{g}}}

\newcommand{\xhat}{\hat{x}}
\newcommand{\rhat}{\hat{r}}
\newcommand{\shat}{\hat{s}}
\newcommand{\Khat}{\hat{K}}
\newcommand{\Bhat}{\hat{B}}
\newcommand{\Ahat}{\hat{A}}
\newcommand{\Dhat}{\hat{D}}
\newcommand{\Shat}{\hat{S}}
\newcommand{\Ghat}{\hat{G}}

\renewcommand{\P}{\mathrm{P}}
\newcommand{\DP}{\mathrm{DP}}
\newcommand{\CG}{\mathrm{CG}}
\newcommand{\DG}{\mathrm{DG}}
\newcommand{\Ned}{\mathrm{Ned}^{1}}
\newcommand{\RT}{\mathrm{RT}}
\newcommand{\BDM}{\mathrm{BDM}}
\newcommand{\NedTwo}{\mathrm{Ned}^{2}}


\newcommand{\Nedelec}{Ned\'el\'ec~}

%\newtheorem{theorem}{Theorem}
\newtheorem{remark}{Remark}


%------------------------------------------------------------------------------
% Document
%------------------------------------------------------------------------------

\title{
   Fast solvers for the high-order FEM simplicial de Rham complex
\thanks{Submitted to the editors July XX, 2024.
      \funding{
         PDB and PEF were supported by EPSRC grant EP/W026260/1.
      }
   }
}

\author{Pablo D.\ Brubeck\thanks{
Mathematical Institute,
University of Oxford,
Oxford, UK (\email{brubeckmarti@maths.ox.ac.uk})
%\orcid{0000-0002-3824-0080}
}
\and Patrick E.\ Farrell\thanks{
Mathematical Institute,
University of Oxford,
Oxford, UK (\email{patrick.farrell@maths.ox.ac.uk})
%\orcid{0000-0002-1241-7060}
}
\and Robert C.\ Kirby\thanks{
Department of Mathematics,
Baylor University,
Waco, TX, USA (\email{robert\_kirby@baylor.edu})
%\orcid{0000-0002-1241-7060}
}
}

\headers{Patch solvers for simplicial $p$-FEM}{P.~D.~Brubeck and P.~E.~Farrell and R.~C.~Kirby}

\begin{document}

\numberwithin{equation}{section}
\maketitle

\begin{abstract}
We present new high-order finite elements discretizing the $L^2$ de Rham
complex on triangular and tetrahedral meshes. The finite elements discretize
the same spaces as usual, but with different basis functions. They allow for
fast linear solvers based on static condensation and space decomposition
methods. The new elements build upon the definition of degrees of freedom
given in [L.~Demkowicz, P.~Monk, L.~Vardapetyan, and W.~Rachowicz,
Comput.~Math.~Appl.,
39(7-8) (2000), pp.~29--38], and consist of integral moments
on an equilateral reference simplex with respect to a numerically computed
polynomial basis that is orthogonal in both the $L^2$- and
$H(\mathrm{d})$-inner products ($\mathrm{d} \in \{\mathrm{grad},
\mathrm{curl}, \mathrm{div}\}$). 
On the reference equilateral simplex, the
resulting stiffness matrix has diagonal interior block, and does not couple
together the interior and interface degrees of freedom. Thus, on the
reference simplex, the Schur complement resulting from elimination of
interior degrees of freedom is simply the interface block itself. This
sparsity is not preserved on arbitrary cells mapped from the reference cell.
Nevertheless, the interior-interface coupling is weak because it is only
induced by the geometric transformation. We devise a preconditioning
strategy by neglecting the interior-interface coupling. We precondition the
interface Schur complement with the interface block, and simply apply
point-Jacobi to precondition the interior block. The combination of this
approach with a space decomposition method on small subdomains constructed
around vertices, edges, and faces allows us to efficiently solve the
canonical Riesz maps in $H(\mathrm{grad})$, $H(\mathrm{curl})$, and
$H(\mathrm{div})$, at very high order. We empirically demonstrate iteration
counts that are robust with respect to the polynomial degree. 
\end{abstract}

\begin{keywords}
   preconditioning, de Rham complex, simplex, high-order, additive Schwarz
\end{keywords}

\begin{AMS}
   65F08, 65N35, 65N55
\end{AMS}

%\begin{DOI}
%10.1137/21XXXXXXX
%\end{DOI}

\section{Introduction} \label{sec:introduction}



\section{Sparsity-promoting discretizations} \label{sec:dofs}

We discretize $\Ltwo$-de Rham complex in triangular and tetrahedral meshes. 

\begin{figure}[htbp] 
\centering
\begin{tikzcd}
  \Hgrad \arrow[r, "\grad"] \arrow[d]  & \Hcurl \arrow[r, "\curl"]
  \arrow[d] & \Hdiv \arrow[r, "\div"]  \arrow[d] & \Ltwo  \arrow[d] \\
  \CG_p \arrow[r, "\grad"] & \Ned_p \arrow[r, "\curl"] & 
   \RT_p \arrow[r, "\div"] & \DG_{p-1}
\end{tikzcd}

\begin{tikzcd}
  \Hgrad \arrow[r, "\grad"] \arrow[d]  & \Hcurl \arrow[r, "\curl"]
  \arrow[d] & \Hdiv \arrow[r, "\div"]  \arrow[d] & \Ltwo  \arrow[d] \\
   \CG_p \arrow[r, "\grad"] & \NedTwo_{p-1} \arrow[r, "\curl"] & 
   \BDM_{p-2} \arrow[r, "\div"] & \DG_{p-3}
\end{tikzcd}
\caption{The $\Ltwo$-de Rham complex, and the finite element subcomplexes of the first and second kind.}
\end{figure}


We follow the definition of interpolation degrees of freedom from
\cite{demkowicz00} to construct dual bases that promote orthogonality in the
$\Ltwo(\Khat)$ and $H(\mathrm{d}, \Khat)$ inner products.

We denote the space of polynomials with vanishing boundary trace as
$\mathbb{B}_{p}(\mathrm{d}, K) = [\P_{p}(K)]^d \cap H_0(\mathrm{d}, K)$.

\subsection{Sparsity-promoting basis for $\Hgrad$}

Let $\Delta^d$ be the equilateral $d$-simplex.
We define a basis for the dual of $\CG_p$ as 
point evaluations at $V \in \text{vertices}(\Delta^d)$,
\begin{equation}
   \ell^V(v) = v(\bx_V),
\end{equation}
and for each sub-entity $S \in \text{edges}(\Delta^d) \cup \text{faces}(\Delta^d) \cup \text{interior}(\Delta^d)$
we take integral moments of surface gradients, 
\begin{equation}
   \ell^S_j(v) = (\grad_S\phi^S_j, \grad_S v)_{S},
\end{equation}
where $\grad_S$ is the tangential gradient on $S$,
and $\{\phi^S_j\}$ is a basis for $\mathbb{B}_{p}(\grad, S)$, such that
\begin{equation} \label{eq:hgrad-fdm}
   (\grad_S\phi^S_j, \grad_S\phi^S_i)_{S} = \delta_{ij}, \quad
   (\phi^S_j, \phi^S_i)_{S} = \lambda_j\delta_{ij}.
\end{equation}
The eigenbases $\{\phi^S_j\}$ are numerically computed offline and only once 
on the reference interval, triangle, and tetrahedron.


\subsection{Sparsity-promoting basis for $\Hcurl$}

Recall the definition of the $\Hcurl$-confroming finite element spaces
\begin{equation}
V^1_{p, p_E}(K) = \{v\in [\P_p(K)]^d : v\cdot\bt \in \P_{p_E}(E) \}
\end{equation}
for $p_E=p-1$ we obtain the \Nedelec space of the first kind $\Ned_p(K)$, while
$p_E=p$ gives the \Nedelec space of the second kind $\NedTwo_p(K) = [\P_p(K)]^d$. 
These spaces enforce continuity of the tangential trace across mesh interfaces.


We define a basis for the dual of $V^1_{p, p_E}(K)$ as 
tangential moments along $E\in \text{edges}(\Delta^d)$,
\begin{equation}
   \ell^E_j(v) = (q_j, v\cdot \bt)_E, \quad q_j \in \P_{p_E}(E), 
\end{equation}
and for each sub-entity $S \in \text{faces}(\Delta^d) \cup \text{interior}(\Delta^d)$,
\begin{align}
   \ell^{S,0}_j(v) &= (\grad_S\phi^S_j, v)_{S}, \\
   \ell^{S,1}_j(v) &= (\curl_S\Phi^S_j, \curl_S v)_{S},
\end{align}
where $\curl_S$ is the tangential curl on $S$,
$\{\phi^S_j\}$ are the $H_0(\grad, S)$ bases constructed in \eqref{eq:hgrad-fdm},
and $\{\curl_S\Phi^S_j\}$ is a basis for $\curl_S \mathbb{B}_{p_S}(\curl, S)$
such that
\begin{equation} \label{eq:hcurl-fdm}
   (\curl_S\Phi^S_j, \curl_S\Phi^S_i)_{S} = \delta_{ij}, \quad
   (\Phi^S_j, \Phi^S_i)_{S} = \lambda_j\delta_{ij}, \quad
   \Phi^S_j\times\bn = 0 \text{ on }\partial S.
\end{equation}


\subsection{Sparsity-promoting basis for $\Hdiv$}

Recall the definition of the $\Hdiv$-confroming finite element spaces
\begin{equation}
V^2_{p, p_F}(K) = \{v\in [\P_p(K)]^d : v\cdot\bn \in \P_{p_F}(F) \}
\end{equation}
for $p_F=p-1$ we obtain the Raviart--Thomas space $\RT_p(K)$, while
$p_F=p$ gives the Brezzi--Douglas--Marini space $\BDM_p(K) = [\P_p(K)]^d$. 
These spaces enforce continuity of the normal trace across mesh interfaces.

We define a basis for the dual of $V^2_{p, p_F}(K)$ as normal moments on $F\in
\text{faces}(\Delta^d)$,
\begin{equation}
   \ell^F_j(v) = (q_j, v\cdot \bn)_F, \quad q_j \in \P_{p_F}(F),
\end{equation}
and on the $\text{interior}(\Delta^d) = K$: 
\begin{align}
   \ell^{K,0}_j(v) &= (\curl\Phi^K_j, v)_{K}, \\
   \ell^{K,1}_j(v) &= (\div\Psi^K_j, \div v)_{K},
\end{align}
where $\{\Phi^K_j\}$ are the $H_0(\curl, S)$ bases constructed in
\eqref{eq:hcurl-fdm}, and $\{\div\Psi^K_j\}$ is a basis for $\div
\mathbb{B}_p(\div, K)$ such that
\begin{equation} \label{eq:hdiv-fdm}
   (\div\Psi^K_j, \div\Psi^K_i)_K = \delta_{ij}, \quad
   (\Psi^K_j, \Psi^K_i)_{K} = \lambda_j\delta_{ij}, \quad
   \Psi^K_j\cdot\bn = 0 \text{ on }\partial K.
\end{equation}


\section{Space decompositions}


We follow 
the low-order space decompositions studied in \cite{arnold00}
and extend them to high-order with static-condensation.

For 0-forms we construct patches on vertex-stars,
\begin{equation}
   V^0_{h,p} = \left.V^0_{h,p}\right|_\mathcal{I} + V^0_{h,1} 
   + \sum_{v\in E^0(\mesh)} \left.\tilde{V}^0_{h,p}\right|_{\star v} 
\end{equation}

For 1-forms we construct edge-star patches
plus the gradient of vertex-star patches on 0-forms
\begin{equation}
   V^1_{h,p} = \left.V^1_{h,p}\right|_\mathcal{I} +  V^1_{h,1}
   + \sum_{e\in E^1(\mesh)} \left.\tilde{V}^1_{h,p}\right|_{\star e} 
   + \sum_{v\in E^0(\mesh)} \mathrm{d} \left.\tilde{V}^0_{h,p}\right|_{\star v} 
\end{equation}

For 2-forms we construct edge-star patches,
\begin{equation}
   V^2_{h,p} = \left.V^2_{h,p}\right|_\mathcal{I} +  V^2_{h,1} 
   + \sum_{e\in E^1(\mesh)} \left.\tilde{V}^2_{h,p}\right|_{\star e}. 
\end{equation}



\section{Numerical results} \label{sec:results}

\subsection{Unit cube tests}
Here we want to address questions about $h$ and $p$ robustness of the space
decomposition. Also how does this method compare with other standard bases in
terms of not only of runtime, memory, and FLOPs, but also conditioning, and
approximation error.

\subsection{Mixed Poisson}
Is there a substantial advantage in the first order reformulation of Poisson solved
with a block Riesz map preconditioner over the $C^0$ formulation?
Since the normal trace of the $\Hdiv$ space is just a scalar, we expect that the
statically-condensed edge patches should couple fewer cells.


\subsection{Complicated geometries and adaptive mesh refinement}
How does the solver perform on complicated and adaptively-refined meshes?


\section{Conclusion} \label{sec:conclusion}



\bibliographystyle{siamplain}
\bibliography{references}
\end{document}
